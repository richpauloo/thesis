\chapter[Introductory Remarks.]{Introductory Remarks}

\section{Background and Motivation}

In 1798 the economist Thomas Malthus speculated that world food demand would outstrip population and lead to the collapse of civilization \citep{malthus1992malthus}. The so-called ``Malthusian curse'' was famously averted by mid-twentieth century scientific innovations that have came to be known as the ``Green Revolution,'' and comprised developments in plant genetics, nitrogen fixation, and the development of groundwater resources for irrigation. Importantly, the development of drilling technology in the past century has allowed humans to access ever-deeper aquifers that, coupled with surface water diversions, have transformed arid and semi-arid regions worldwide into productive bread baskets. Today, groundwater provides part of the drinking water supply for at least 50\% of the global population, and accounts for 43\% of all water used for irrigation \citep{siebert2010groundwater}. It is estimated that 2.5 Billion people--about one in three people on Earth--depend solely on groundwater resources to satisfy their basic daily water needs \citep{unesco2015}. However, despite its critical role in sustaining food production regions of the world and providing basic drinking water needs to a substantial fraction of the human population, groundwater resources are imperiled. Numerous studies have documented the long-term unsustainable extraction of groundwater \citep{Famiglietti2014, wada2010global, Doll2012, Gleeson2012, de2019environmental}, and numerous others have documented the negative consequences of groundwater pumping, including land subsidence \citep{Brush2013}, leeching of contaminants and other impacts to subsurface contaminant migration \citep{smith2018overpumping}, chronic decline of groundwater level and storage \citep{Scanlon2012}, surface water depletion and loss of groundwater dependent ecosystems \citep{TNC2014}, and seawater intrusion \citep{bear1999seawater}. These challenges have been met by the development of groundwater flow and contaminant transport models \citep{domenico1998physical, Fetter2001}, remote sensing and GPS methods that measure gravimetric fluctuations \citep{ramillien2008detection} and land surface deformation \citep{galloway1998detection}, and geophysical methods \citep{goebel2019mapping}. Nonetheless, ongoing groundwater depletion has created new and emerging consequences of regional scale aquifer depletion, including well failure and regional-scale groundwater basin salinization, and these understudied challenges require the development of methods and models to adequately describe, understand, and forecast the severity of their impact to groundwater supplies.  

In California alone, more than half of all pumped groundwater is used for irrigation \citep{Faunted.2009}, about 85\% of Californians depend on groundwater for some portion of their water supply \citep{ppic2017gw}, and 1.5 million residents rely solely on domestic wells for drinking water \citep{Dieter2018}. Groundwater development in the past century has transformed California's Central Valley into one of the most heavily irrigated and economically productive agricultural regions in the world, and also made it a case study in chronic historical overdraft \citep{Hanak2011}. In the backdrop of these groundwater challenges in California (which are emblematic of global groundwater challenges) is a warming climate, and it is in this increasingly drier and warmer climate \citep{Diffenbaugh2015, Cook2015} characterized by more frequent, more spatially extensive heat waves and extended droughts \citep{Tebaldi2006, Lobell2011}, that California will implement a statewide policy of sustainable groundwater management \citep{SGMA}. These policies aim to prevent chronic groundwater overdraft and other undesirable results, but fail to anticipate two critical challenges: domestic well failure and regional groundwater salinization due to basin closure induced and then sustained by pumping. Presently, we lack conceptual models and methodological approaches to describe and anticipate these critical challenges, and thus, inform the implementation of policy and mitigative actions aimed minimizing the harm caused by them. It is the aim of this dissertation to address the knowledge gaps surrounding well failure and closed basin salinization.


\section{Objectives}

In this dissertation, data-driven and numerical models of well failure, basin salinization, and contaminant transport are developed in order to address exigent and emerging threats to fresh groundwater resources in a typical alluvial aquifer-aquitard system, using California's Central Valley as a case study. The specific objectives of this dissertation are:


\begin{itemize}
	
    \item Develop a data-driven, regional-scale model of well failure of California's Central Valley that \textit{(i)} reproduces the well failure impacts of the historical severe 2012-2016 drought, and forecasts the impact of: \textit{(ii)} extended (5 to 8 year long) drought duration scenarios, \textit{(iii)} three different groundwater management regimes (sustainability, glide path, business as usual), and \textit{(iv)} the impact of wet winter recharge events. (Chapter 2)
    
    \item Develop a conceptual framework and simple mixing tank model to estimate the approximate timescales and depth scales associated with regional aquifer salinization in California's Tulare Basin. (Chapter 3)
    
    \item Extend the work presented in Chapter 3 by using numerical groundwater flow and contaminant transport modeling to investigate how mean flow direction and hydraulic boundary condition transience modulate the degree of non-Fickian transport in a typical alluvial aquifer-aquitard system. This study aims to improve our understanding of the various methods to upscale transport, and inform the development of new upscaling methods. (Chapter 4)
   
    
\end{itemize}