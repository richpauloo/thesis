% Declarations for Front Matter

% MS Thesis = 0, Phd Dissertation = 1
\isdissertation{1}

% electronic submission? Paper only = 0, Electronic = 1
\iselectronic{1}

%\title{TEST}
\title{Emerging consequences of regional-scale aquifer depletion: data-driven and numerical models of well failure, basin salinization, and contaminant transport}

\author{Richard Augustus Pauloo III}

% Choices are September, December, March, June
\degreemonth{September}
\degreeyear{2020}
%More examples DOCTOR OF PHILOSOPHY
\degree{DOCTOR OF PHILOSOPHY}

\chair{Graham E Fogg}
\othermembers{Thomas Harter, Jonathan Herman} %comma separated list of committee not including chair
\numberofmembers{3} % size of committee

%\prevdegrees{B.S. (University of Nevada, Reno) 2010\break M.S. (University of Nevada, Reno) 2012}

%Your Graduate Group
\field{Hydrologic Science}
\campus{Davis}


% add the abstract here
% Their are two abstracts. One that is published externally from your
% dissertation, and one that is internal. Of course, the text of the
% abstract will be the same. So, we define a macro to hold the body of our
% abstract.
% at 345 words - With electronic filing there is no longer a word limit

\newcommand{\myabstract}{

In the past century, global groundwater extraction has transformed arid and semi-arid regions worldwide into areas of significant food production, and provided water for drinking and basic needs to billions of people. However, despite its critical role in irrigated agriculture and drinking water supply, negative consequences of excessive groundwater development threaten the long-term sustainability of major aquifer systems. The numerous consequences of groundwater over-extraction include land subsidence, leeching of contaminants, chronic decline of groundwater level and aquifer storage, surface water depletion, loss of groundwater dependent ecosystems, and seawater intrusion. Emerging challenges of aquifer depletion that have received much less attention include well failure and closed basin salinization. These topics are understudied not because they are unimportant, but rather, because the data and models to describe them have only recently become available, or have not yet been developed. Three case studies using the Central Valley as a study site are presented herein, and these studies advance novel models and conceptual frameworks to address the emerging challenges of well failure and closed basin salinization. First a data-driven model of well failure forecasts the magnitude and spatial distribution of domestic well failure resulting from extended (5 to 8 years in length) drought scenarios, and show that groundwater management regimes (business as usual, glide path, strict sustainability), and wet winter recharge events substantially influence the occurrence of well failure. Next a conceptual model and first-order estimates of the timescales and depth scales of groundwater salinization due to basin closure in California's Tulare Basin are presented. Results suggest that shallow aquifer salinization proceeds at similar rates to groundwater lifespan in the study area and point towards an alternate groundwater management paradigm in which greater emphasis is placed on subsurface storage in order to keep groundwater basins open, and hence, fresh. Finally, a study in groundwater flow and contaminant transport in the Kings River Fan shows that varying the mean flow direction in a 3D hydrofacies model can lead to differing degrees of non-Fickian transport, with implications for efforts to upscale transport modeling of nonpoint source plumes, and the conceptualization of transport in highly heterogeneous alluvial aquifer-aquitard systems. 

}



%Not required for electronic submission, you will need to print your other abstract page 2x and hand them in.
% Here is the first, external, abstract.
%\begin{abstract}
%	\myabstract
%	\abstractsignature
%\end{abstract}

\begin{frontmatter}
\maketitle

% A copyright page is optional. If you have one, it must immediately
% follow the title page. For more information about the copyright page
% see the UCD's Office of Graduate Studies web site.
\copyrightpage

% dedication (optional), remove comment markers to use 
%\begin{dedication}
%\null\vfil
%{\large
%\begin{center}
%xxxx
%\end{center}}
%\vfil\null
%\end{dedication}

\tableofcontents
\listoffigures
\listoftables

% Here is the second, internal, abstract.
% Update: Melissa Danforth 2006
% Inline abstract is now part of front matter according to coordinator
    \newpage
    \begin{inlineabstract}
		%Only enable small if you're trying to make it fit.		
		%\begin{small}
		\myabstract
		%\end{small}
		
    \end{inlineabstract}

%Acknowledgments (optional)
\begin{acknowledgments}
	
The Chinese philosopher Chuang Tzu wrote, ``Your life has a limit, but knowledge has none. If you use what is limited to pursue what has no limit, you will be in danger.'' 

This work therefore, is the product of living dangerously. I've spent many years of my life--time I will never get back--in pursuit of knowledge. Some of my most cherished moments have been precisely when I've grasped a sense of how infinite knowledge can feel, where a mind is a drop in an ocean, how the currents of knowledge expand like an inflationary universe, how knowledge accretes from the collective effort of people buried in books, clamoring over computers, laboring in labs, exploring the environment, studying social structures. This small dissertation are three drops of water in an ever-growing ocean of knowledge, three drops of water that I've been honored and humbled to document, three drops of water that I could not have described or understood were it not for a community of people lifting me up, a community of people that have my most sincere and heartfelt appreciation. 

I am deeply grateful for my adviser Graham Fogg. Graham gave me the freedom to make mistakes, to learn how to think independently, and to develop scientific and intellectual curiosity. I also want to thank my dissertation committee members Thomas Harter, Laura Foglia, Jon Herman, Randy Dahlgren, and Robert Hijmans for sharing their ideas with me and steering me in the right direction. I especially want to thank Jon Herman and Robert Hijmans for instilling a love of programming through their obstinate proselytization of Python and R--those courses forever changed how I approach scientific problems and I'm fully converted. Thank you to close collaborators who have helped me along the way, in particular Zhilin Guo, Christopher Henri, Alvar Escriva-Bou, Helen Dahlke, Andrew Calderwood, Darcy Bostic, Herve Guillon, and Amanda Fencl. Aakash Ahmed, Corey Scher, and Kyra Kim have been a tremendous source of inspiration for me--meeting them reassured me that I was on the right path in life. Each of these scientists has selflessly taken the time to educate me, share their ideas and refine mine, and contribute to my work. I have nothing but gratitude, admiration, and respect for these mentors, collaborators, and friends.

I'm also very thankful for the support of the Guardian Professions Program, specifically the indefatigable and generous Sylvia Sensiper who coached me more than once on the mysteries of graduate school, and who helped fund my education. Carole Hom deserves special thanks for her dedication to training an interdisciplinary group of scientists, which I was fortunate to be a part of. Shila Ruiz always had an open door, an open mind, and a smile to share. Sam Sandoval has been a warm and supportive academic advisor and co-conspirator. 

Many groups have funded my research, travel, and training, including the National Science Foundation Climate Change Water and Society (CCWAS) IGERT, the University of California Water (UC Water) Security and Sustainability Research Initiative, the US Department of Energy's US-China Center for Energy and Clean Energy Research Center for Water-Energy Technologies (CERC-WET), the American Geophysical Union, the West Big Data Hub, NASA, Microsoft, and RStudio. 

My labmates and the greater HSGG community have been a source of friendship, strength, and inspiration throughout the years. Students in the program have enriched my academic and personal life, and in particular I want to thank Katie Markovich, Steve Maples, Gus Tolley, Noelle Ptterson, Jason Weiner, and Marina Mautner for our conversations and camaraderie. Looking forward, the newer HSGG students continue to inspire me, and I'm glad to have overlapped with many, especially Bill Rice, Claire Kouba, Nusrat Molla, Bradley Simms, and Sam Winter.

Exchanges with water policy experts over the years have also transformed how I think about the role of science in society, and I want to thank Jeff Mount, Ellen Hanak, Meredith Lee, Debbie Franco, Greg Gearheart, and Brent Vanderburgh for those energizing dialogues.

There are hundreds, perhaps thousands of people who have directly or indirectly helped me solve coding problems on Stack Overflow, at the UC Davis Data Lab, and within the UC Davis geospatial and R-users group forums. These communities of practice have sustained me when I've hit walls, and selflessly shared their time to help grow and tackle increasingly challenging problems.

To Natalie Popovich and Julia Michaels, my first friends in Davis and housemates of nearly 3 years, thank you for your friendship and humor, for celebrating achievements with me, and for routinely turning our house in a classical music concert hall. These are among my most cherished memories of this sleepy college town. To Ben Dawson, Lillie Mansfield, Bree DeRobbio, and Grace Woodmansee thank you for keeping me grounded, for group meals, and for sharing your pace of life. To Gabe Patterson, Meghan Jones, John Mitchell, Jess Gold, Jake Stahl, and Jenn Nill thank you for being my COVID-19 pod, for trips to the mountains, and for getting me through the final days of my dissertation. 

Three powerful women deserve special recognition, and those women are my mother, sister, and partner. If this dissertation is dedicated to anyone, it's to them. I'm often humbled by the sacrifices my mother made to create opportunity for me. I wouldn't be here today if my she had not decided to immigrate from Thailand to the United States, and then, as a single mother, work two jobs to support her children. She is the meaning of resilience. My sister's love, iron-will, and belief in me has helped me persist in the toughest times. Thank you so much for everything Jen. And finally, I could not have done this without the unconditional love and support of my partner Marisa. Thank you for being my rock, my sanity, and my best friend. 





\end{acknowledgments}

%Preface (optional)
\chapter*{Preface}
\addcontentsline{toc}{chapter}{Preface}
\section*{About this dissertation}   

The work that comprises this dissertation is the result of many fruitful collaborations across the University of California system and with other universities, along with myriad other federal, state, and local agencies and NGOs. Chapters 2--4 are my own work, and Chapter 5 was a collaborative effort of Ph.D. students in the CCWAS IGERT at UC Davis and Colorado School of Mines, in which I took a lead role. 

\section*{Notes on Chapter 2}

Chapter 2 was coauthored with Graham Fogg and Reed Maxwell and builds on the previous work of Casey Meirovitz and Yunjie Liu. The content emerged from discussions with all of the coauthors (and our colleagues) on the role of geologic heterogeneity on managed aquifer recharge (MAR) processes. All of the modeling was performed on Cheyenne, the National Center for Atmospheric Research's (NCAR) high-performance computer. This work has been accepted in \textit{Hydrogeology Journal} (doi: 10.1007/s10040-019-02033-9) and will likely appear in that journal in Fall 2019.
   
\section*{Notes on Chapter 3} 

Chapter 3 builds on the work from Chapter 1 and was coauthored with Laura Foglia, Graham Fogg, and Reed Maxwell. This work explores unanswered questions from the MAR simulations in Chapter 2 using novel sensitivity analysis techniques. Namely, we evaluate the relative importance of geologic architecture on MAR processes, as compared with other model parameters. All modeling was also performed on NCAR Cheyenne. This work is in review at \textit{Hydrologic and Earth System Science} (\textit{HESS}; preprint doi: 10.5194/hess-2019-412) and will likely appear in that journal in early 2020. 
    
\section*{Notes on Chapter 4} 

Chapter 4 compares the methodology for estimating water budgets, and especially unmeasured agricultural groundwater pumping, in the Central Valley of California. This work is coauthored by Graham Fogg, Laura Foglia, and Thomas Harter and emerged, in part, with the passage of the Sustainable Groundwater Management Act (SGMA) in 2014, which mandates that California better manage its groundwater resources. This work is in preparation for submission to \textit{Journal of Water Resources Planning and Management}. Portions of this research have also been contributed to a manuscript recently submitted to the \textit{Journal of Environmental Management}, for which I am a coauthor.

\section*{Notes on Chapter 5} 

Chapter 5 is an opinion piece which outlines how the combination of comprehensive water resources accounting and a functioning water market that includes a central clearinghouse for water transactions holds great promise for reducing magical thinking and dissolving many water resources management obstacles. I was was the lead author of this effort and was responsible for the bulk of the writing; however, I am deeply indebted to each of my coauthors, Ellen Bruno, Alejo Kraus-Polk, Stacy Roberts, and Lauren Foster, for their distinct perspectives and contributions to this piece. The concept for this manuscript emerged from the authors’ varied disciplines (hydrology, history, economics, geography) and from a conference that we and other CCWAS students developed and held in April, 2015 at the University of California, Davis. The conference, \textit{Water Scarcity in the West: Past, Present, Future}, brought together a diverse group of scholars to address the issue of water scarcity globally and through time. This work was published in a special issue of the journal \textit{Water} in late 2018 (doi: 10.3390/w10121720).


\end{frontmatter}
