% Declarations for Front Matter

% MS Thesis = 0, Phd Dissertation = 1
\isdissertation{1}

% electronic submission? Paper only = 0, Electronic = 1
\iselectronic{1}

%\title{TEST}
\title{Emerging consequences of regional-scale aquifer depletion: data-driven and numerical models of well failure, basin salinization, and contaminant transport}

\author{Richard Augustus Pauloo III}

% Choices are September, December, March, June
\degreemonth{September}
\degreeyear{2020}
%More examples DOCTOR OF PHILOSOPHY
\degree{DOCTOR OF PHILOSOPHY}

\chair{Graham E Fogg}
\othermembers{Thomas Harter, Jonathan Herman} %comma separated list of committee not including chair
\numberofmembers{3} % size of committee

%\prevdegrees{B.S. (University of Nevada, Reno) 2010\break M.S. (University of Nevada, Reno) 2012}

%Your Graduate Group
\field{Hydrologic Science}
\campus{Davis}


% add the abstract here
% Their are two abstracts. One that is published externally from your
% dissertation, and one that is internal. Of course, the text of the
% abstract will be the same. So, we define a macro to hold the body of our
% abstract.
% at 345 words - With electronic filing there is no longer a word limit

\newcommand{\myabstract}{

Civilizations have typically obtained water from surface-water resources throughout most of human history. Only during the last 50--70 years has a significant quantity of water been obtained by pumping groundwater. During this short time, alarming levels of groundwater depletion have been observed worldwide, especially in semi-arid and arid regions that rely heavily on groundwater pumping from clastic sedimentary basins. In order to reverse the negative effects of over-exploitation of groundwater resources, we must transition from treating groundwater mainly as an extractive resource to one in which sustainable management is pursued more aggressively. Numerical groundwater models are valuable tools for characterizing groundwater systems, but are often hamstrung by limited data and/or poor process representation. Several case studies are presented to highlight both the value and the uncertainty of models for understanding water budgets and representing water-flow processes. Simulations of managed aquifer recharge (MAR) dynamics reveal the influence of alluvial geologic architecture on MAR potential and highlight the value of capturing detailed geologic heterogeneity and physical processes that are not typically included in groundwater models when evaluating MAR potential. Additionally, a comparison of two regional-scale agricultural groundwater models of the Central Valley shows how both conceptual differences can contribute to differences in water budget estimates when there is a lack of comprehensive groundwater pumping measurements. Finally, an opinion is presented which outlines how the combination of comprehensive water resources accounting and a functioning water market that includes a central clearinghouse for water transactions holds great promise for reducing magical thinking and dissolving many water resources management obstacles. 
}


%Not required for electronic submission, you will need to print your other abstract page 2x and hand them in.
% Here is the first, external, abstract.
%\begin{abstract}
%	\myabstract
%	\abstractsignature
%\end{abstract}

\begin{frontmatter}
\maketitle

% A copyright page is optional. If you have one, it must immediately
% follow the title page. For more information about the copyright page
% see the UCD's Office of Graduate Studies web site.
\copyrightpage

% dedication (optional), remove comment markers to use 
%\begin{dedication}
%\null\vfil
%{\large
%\begin{center}
%xxxx
%\end{center}}
%\vfil\null
%\end{dedication}

\tableofcontents
\listoffigures
\listoftables

% Here is the second, internal, abstract.
% Update: Melissa Danforth 2006
% Inline abstract is now part of front matter according to coordinator
    \newpage
    \begin{inlineabstract}
		%Only enable small if you're trying to make it fit.		
		%\begin{small}
		\myabstract
		%\end{small}
		
    \end{inlineabstract}

%Acknowledgments (optional)
\begin{acknowledgments}
	
The Chinese philosopher Chuang Tzu wrote, ``Your life has a limit, but knowledge has none. If you use what is limited to pursue what has no limit, you will be in danger.'' 

This work therefore, is the product of living dangerously. I've spent many years of my life--time I will never get back--in pursuit of knowledge. Some of my most cherished moments have been precisely when I've grasped a sense of how infinite knowledge can feel, where a mind is a drop in an ocean, how the currents of knowledge expand like an inflationary universe, how knowledge accretes from the collective effort of people buried in books, clamoring over computers, laboring in labs, exploring the environment, studying social structures. This small dissertation are three drops of water in an ever-growing ocean of knowledge, three drops of water that I've been honored and humbled to document, three drops of water that I could not have described or understood were it not for a community of people lifting me up, a community of people that have my most sincere and heartfelt appreciation. 

I am deeply grateful for my adviser Graham Fogg. Graham gave me the freedom to make mistakes, to learn how to think independently, and to develop scientific and intellectual curiosity. I also want to thank my dissertation and qualifying exam committee members Thomas Harter, Laura Foglia, Jon Herman, Randy Dahlgren, and Robert Hijmans for sharing their ideas with me and steering me in the right direction. I especially want to thank Jon Herman and Robert Hijmans for instilling a love of programming through their obstinate proselytization of Python and R--those courses forever changed how I approach scientific problems and I'm fully converted. Thank you to close collaborators who have helped me along the way, in particular Zhilin Guo, Christopher Henri, Alvar Escriva-Bou, Helen Dahlke, Andrew Calderwood, Darcy Bostic, Herve Guillon, and Amanda Fencl. Aakash Ahmed, Corey Scher, and Kyra Kim have been a tremendous source of inspiration for me--meeting them reassured me that I was on the right path in life. Each of these scientists has selflessly taken the time to educate me, share their ideas and refine mine, and contribute to my work. I have nothing but gratitude, admiration, and respect for these mentors, collaborators, and friends--my scientific community.

I'm also very thankful for the support of the Guardian Professions Program, specifically the indefatigable and generous Sylvia Sensiper who coached me more than once on the mysteries of graduate school, helped fund my education, and encouraged me to think bigger about my ambitions long before I believed in myself by saying, ``I meet a lot of prospective students, and you seem like PhD material to me''. Carole Hom deserves special thanks for her dedication to training an interdisciplinary group of scientists, which I was fortunate to be a part of and learn from. Shila Ruiz is a saint: she always had an open door, an open mind, and a smile to share. Sam Sandoval has been a warm and supportive academic advisor and co-conspirator. 

Many groups have funded my research, travel, and training, including the National Science Foundation Climate Change Water and Society (CCWAS) IGERT, the University of California Water (UC Water) Security and Sustainability Research Initiative, the US Department of Energy's US-China Center for Energy and Clean Energy Research Center for Water-Energy Technologies (CERC-WET), the American Geophysical Union, the West Big Data Hub, NASA, Microsoft, and RStudio. 

My labmates, the HSGG community, and the greater UC Davis graduate student cohort have been a source of friendship, strength, and inspiration throughout the years. Students in the program have enriched my academic and personal life and transformed this quiet college town into a supportive backdrop for intellectual exploration and growth. In particular I want to thank Jason Wiener, Steve Maples, Gus Tolley, Gaby Castrellon, Amy Yoder, Katie Markovich, Noelle Patterson, Kim Miles, Tara Seely, Bal Sah, Sam Sharp, Lisa Rosenthal, Ernst Bo, Ali Hill, Angadh Nanjangud, and Marina Mautner for our conversations and camaraderie. Looking forward, the newer HSGG students continue to inspire me and teach me new things, and I'm glad to have overlapped with many, including Bill Rice, Claire Kouba, Nusrat Molla, Bradley Simms, and Sam Winter.

Exchanges with water policy experts over the years have also transformed how I think about the role of science in society, and I want to thank Jeff Mount, Ellen Hanak, Meredith Lee, Debbie Franco, Greg Gearheart, and Brent Vanderburgh for our energizing dialogues.

There are hundreds, perhaps thousands of anonymous people who have directly or indirectly helped me solve coding problems on Stack Overflow, at the UC Davis Data Lab, and within the UC Davis geospatial and R-users group forums. These communities of practice have sustained me when I've hit walls, and selflessly shared their time to help me grow and tackle increasingly challenging problems.

To Natalie Popovich and Julia Michaels, my first friends in Davis and housemates of nearly 3 years, thank you for your friendship and humor, for celebrating achievements with me, and for routinely turning our house in a classical music concert hall. These are among my most cherished memories of this sleepy college town. To Ben Dawson, Lillie Mansfield, Bree DeRobbio, and Grace Woodmansee thank you for keeping me grounded, for group meals, and for sharing your perspectives and pace of life. To Gabe Patterson, Meghan Jones, John Mitchell, Jess Gold, Jake Stahl, and Jenn Nill thank you for being my COVID-19 pod, for trips to the mountains, and for getting me through the final days of my dissertation. 

Three powerful women deserve special recognition, and those women are my mother, sister, and partner. If this dissertation is dedicated to anyone, it's to them. I'm often humbled by the sacrifices my mother made to create opportunity for me. I wouldn't be here today if my she had not decided to immigrate from Thailand to the United States, and then, as a single mother, work two jobs to support her children. She is the meaning of resilience. My sister's love, iron-will, and belief in me has helped me persist in the toughest times. Thank you so much for everything Jen. And finally, I could not have done this without the unconditional love and support of my partner Marisa. Thank you for being my rock, my sanity, and my best friend. 





\end{acknowledgments}

%Preface (optional)
\chapter*{Preface}
\addcontentsline{toc}{chapter}{Preface}
\section*{About this dissertation}   

Groundwater resources are found in diverse settings worldwide, and like groundwater, the work herein is the product of many conversations and collaborations between researchers from the US, China, and Europe. 


\section*{Notes on Chapter 2}

Chapter 2 presents a data-driven methodology for estimating well failure applied to California's Central Valley. It was coauthored with Graham Fogg, Alvar Escriva-Bou, Helen Dahlke, Herve Guillon, and Amanda Fencl. The topic of statewide domestic well failure analysis arose after nearly one million digitized well completion reports detailing over 100 years of groundwater development in California were released to the public, permitting the first domestic well failure models. The initial kernel of this chapter won an award at the 2018 California Water Data Challenge, then grew to encompass a wider body of work. This work was published in \textit{Environmental Research Letters} (doi: 10.1088/1748-9326/ab6f10).

   
\section*{Notes on Chapter 3} 

Chapter 3 explores another consequence of aquifer depletion, and that is basin closure and subsequent salinization. Graham Fogg, Zhilin Guo and Thomas Harter were co-authors. This work was been submitted on 2020-07-07 to \textit{Journal of Hydrology} (preprint doi: 10.1002/essoar.10502733.1) and will likely appear in that journal by late 2020 or early 2021. 
    
    
\section*{Notes on Chapter 4} 

Chapter 4 builds on Chapter 3 by exploring the physical and hydrogeologic controls on subsurface contaminant transport that challenge the development of regional, upscaled transport models. It was coauthored with Graham Fogg, Zhilin Guo, and Christopher Henri. This work is in preparation for submission to \textit{Water Resources Research}. 




\end{frontmatter}
