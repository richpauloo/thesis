% Their are two abstracts. One that is published externally from your
% dissertation, and one that is internal. Of course, the text of the
% abstract will be the same. So, we define a macro to hold the body of our
% abstract.
% at 345 words - With electronic filing there is no longer a word limit

\newcommand{\myabstract}{

In the past century, global groundwater extraction has transformed arid and semi-arid regions worldwide into areas of significant food production, and provided water for drinking and basic needs to billions of people. However, despite its critical role in irrigated agriculture and drinking water supply, negative consequences of excessive groundwater development threaten the long-term sustainability of major aquifer systems. The numerous consequences of groundwater over-extraction include land subsidence, leeching of contaminants, chronic decline of groundwater level and aquifer storage, surface water depletion, loss of groundwater dependent ecosystems, and seawater intrusion. Emerging challenges of aquifer depletion that have received much less attention include well failure and closed basin salinization. These topics are understudied not because they are unimportant, but rather, because the data and models to describe them have only recently become available, or have not yet been developed. Three case studies using the Central Valley as a study site are presented herein, and these studies advance novel models and conceptual frameworks to address the emerging challenges of well failure and closed basin salinization. First a data-driven model of well failure forecasts the magnitude and spatial distribution of domestic well failure resulting from extended (5 to 8 years in length) drought scenarios, and shows that groundwater management regimes (business as usual, glide path, strict sustainability), and wet winter recharge events substantially influence the occurrence of well failure. Next a conceptual model and first-order estimates of the timescales and depth scales of groundwater salinization due to basin closure in California's Tulare Basin are presented. Results suggest that shallow aquifer salinization proceeds at similar rates to groundwater lifespan in the study area and point towards an alternate groundwater management paradigm in which greater emphasis is placed on subsurface storage in order to keep groundwater basins open, and hence, fresh. Finally, a study in groundwater flow and contaminant transport in the Kings River Fan shows that varying the mean flow direction in a 3D hydrofacies model can lead to differing degrees of non-Fickian transport, with implications for efforts to upscale transport modeling of nonpoint source plumes, and the conceptualization of transport in highly heterogeneous alluvial aquifer-aquitard systems. 

}

