\chapter[Concluding Remarks.]{Concluding Remarks}

Writing near the turn of the 21st century, Dr. Norman Borlaug, a Nobel Laureate and key figure in the Green Revolution credited with pivotal advancements in plant genetics, noted that the continued success of the Green Revolution depended on fresh water availability, and called for a ``Blue Revolution'' to secure reliable water resources. Civilization's reliance on unsustainable groundwater pumping, especially in clastic sedimentary groundwater basins found in arid and semi-arid regions worldwide, limits the lifespan of these freshwater stores \citep{Scanlon2012, wada2010global, Gleeson2012}. For example, in California's Central Valley, aquifer lifespan has been estimated at 390 years or less depending on the exact location and depletion rate \citep{Scanlon2012}. Although aquifer depletion poses a long-term existential threat, a range of negative consequences (e.g., land subsidence, surface water depletion, destruction of groundwater-dependent ecosystems) will impact surface and groundwater systems long before aquifer stores run dangerously low, and these impacts will be exacerbated by surface water shortages in a warming climate \citep{Rhoades2018, Swain2018, Cook2015}, which has historically increased groundwater pumping to augment lost surface water supply \citep{Hanak2011, Medellin-azuara2016}. Thus, by addressing the short term consequences of aquifer depletion, the larger and slower-moving existential threat of aquifer depletion is also mitigated. 

This dissertation focuses on the development of models to address two understudied consequences of aquifer depletion: well failure and closed basin salinization. Chapter 2 demonstrates how data assimilation from monitoring well networks and open data from state agencies can be combined into a physical model of well failure that forecasts the impact of extended drought duration, groundwater management regimes, and wet winter recharge events in California's Central Valley. Chapter 4 advances a conceptual model of Anthropogenic Basin Closure and groundwater SALinization (ABCSAL), and first-order estimates of salinization in California's Tulare Basin. Chapter 3 uses 3D numerical flow and transport modeling in a complex hydrofacis model to show that transience in the mean flow direction caused by naturally-occurring (due to natural anisotropy in K) and induced hydraulic gradient transience (from pumping and recharge) can change the governing mass transfer processes in hyrdofacies, leading to differing degrees of non-Fickian transport. 


Chapter 2 demonstrates the vulnerability of domestic wells to both naturally-occurring hydrologic events (e.g., drought and wet winter recharge) and human-made water management decisions (e.g., sustainable and business as usual management regimes). Results indicate that extended drought durations (e.g., 5 to 8 years) can lead to a greater annual rate of well failure than observed during historic droughts (e.g., 2012-2016) as groundwater levels intersect an increasingly dense portion of the distribution of domestic well pump depths. Moreover, human water management decisons matter. Four times as many wells failed--even when controlling for hydrologic uncertainty--between a strict sustainability scenario in which groundwater levels were not allowed to fall after 2020, compared to a business as usual scenario that projected current groundwater level decline trends into 2040. Wet winter recharge events, and hence reduced pumping \citep{Hanak2019} and flood-based managed aquifer recharge \citep{Kocis2017} show potential to lessen the degree of domestic well failure. Looking forward, as Internet-of-things technologies mature and remote sensor networks are increasingly cost-effective and easy to maintain, the real-time monitoring of groundwater \citep{calderwood2020low} will permit data-driven models like the one presented in this study to automatically assimilate real-time groundwater level data and provide on-the-fly well failure risk estimates. Thus, it is conceivable that an early warning system for domestic well failure--not unlike existing early warning systems for drought \citep{pozzi2013toward, huang2004drought}--may be a future possibility enabled by sensor technology and cloud computing.

Similarly to Chapter 2, the topic of Chapter 3 remains understudied due to its relatively new emergence. Chapter 3 explores a mechanism by which groundwater pumping eliminates hydrologic exits for naturally-occurring salts in a basin (e.g., via baseflow and lateral subsurface outflow), after which those entrained salts accumulate in shallow aquifers due recycling from pumping and irrigation, before they are driven into deeper aquifer over century-long timescales. We call this process Anthropogenic Basin Closure and groundwater SALinization (ABCSAL). Importantly, results from a first-order mixing cell model indicate that shallow aquifers reach drinking water minimum thresholds for total dissolved solids (TDS) within decades, a result consistent with measurements of shallow aquifer TDS increase in the past century in the study site \citep{Hansen2018}. These results suggest that one mechanism to reverse or slow ongoing ABSCAL may include ``filling the basin up'' with significant recharge to restore baseflow and lateral subsurface outflow. Although the practical reality of such a management action may be cost-prohibitive or physically limited by available surface water in the Tulare Basin, it remains a viable solution in other basins worldwide experiencing ABCSAL. Moreover, a greater emphasis on subsurface water storage will require tightly monitoring and managing the ground and surface water interactions with distributed sensor networks, in order to prevent capillary rise and salinization from bare soil evaporation \citep{belitz1995alternative}. Without these mitigative management actions, ongoing and long-term ABCSAL may require inland desalinization of pumped groundwater.

Building on the work presented in Chapter 3, Chapter 4 uses 3D numerical groundwater flow and transport simulations in a detailed hydrofacies model with large variance in K \citep{weissmann1999multi} to better understand how hydraulic transience in the mean flow direction influences the degree of non-Fickian transport observed in hydrogeologic systems (e.g., nonpoint source salts discussed in Chapter 2). In California's Central Valley and other typical clastic sedimentary alluvial aquifer-aquitard systems worldwide, high vertical anisotropy in K naturally created by the inter-bedding of high and low K sediments, combined with pumping and recharge, can create flow systems that oscillate between predominately vertical and horizontal flow. Although other studies have noted that classical methods to upscale transport fail to represent tailing when the mean flow direction changes \citep{guo2019upscaling, guo2020adaptive}, the hydrogeologic underpinnings of this phenomenon remain poorly understood. Results indicate that relatively higher vertical to horizontal gradient ratio (VHGR) coincident with increasingly vertical mean flow direction can transition low-K facies from diffusion- to advection-dominant, and cause vertical groundwater flow to move directly through facies that typically act as aquitards at lower VHGR (e.g., paleosols, clays, and silts). In other words, when systems transition from diffusion- to advection-dominant, non-Fickian behavior (e.g., tailing, spatial variance, preferential flow) all decrease. These results imply a conceptual model of oscillating patterns in the degree of non-Fickian transport that correspond to seasonal patterns of pumping and recharge. Hence, non-Fickian transport models will better characterize flow systems under higher VHGR, depending on the gradients, characteristic facies length scales, and the Peclet number ($P_e$) of the distinct hydrofacies which describes the ratio of advection to diffusion time scales. These results also suggest a physical mechanism for arsenic leeching from low-K clays observed in the Central Valley during periods of land subsidence \citep{smith2018overpumping}, driven by advective flushing as strong vertical gradients ``force'' groundwater through clays. Lastly, the strong dependence of mass transfer on hydrogeologic features illustrated in this study suggests that future efforts to upscale regional-scale, transient nonpoint source transport may benefit by incorporating information on the time-dependent changes in VHGR, $P_e$, mean flow direction, and characteristic length scales along the direction of mean flow.

Securing sustainable groundwater resources in the 21st century and beyond will require the further development and refinement of hydrogeologic science, and numerical and data-driven models to assimilate that understanding and provide critical insights. Although this dissertation focuses on groundwater problems in California, the themes explored herein are extensible to other heavily pumped groundwater systems worldwide. Indeed, it is only through enhanced understand of aquifers that civilization may bring about a Blue Revolution in groundwater resources monitoring, modeling, and management.

\clearpage