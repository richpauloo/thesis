\chapter[Concluding Remarks.]{Concluding Remarks}

The widespread adoption of the submersible turbine pump in the early- to mid-20th century brought about an explosion in the reliance on groundwater worldwide \citep{alley2002flow}. During this time, unsustainable groundwater depletion has been observed in many semi-arid and arid areas that rely heavily on groundwater pumping from clastic sedimentary basins \citep{konikow2013groundwater,taylor2013ground,wada2010global,wada2012nonsustainable} and has only accelerated in many places during the 21st century \citep{wada2011modelling}. Groundwater is typically treated as a backstop during periods of scarcity \citep{hanson2012method}, and is becoming increasingly important as climate change threatens existing surface water sources \citep{milly2008stationarity,mirchi2013climate}.

Broadly, this dissertation presents is a framework for assessing groundwater availability and sustainability in agriculturally-dominated regions like California's Central Valley. Numerical groundwater models were used to evaluate the feasibility of groundwater recharge, assess questions related to model complexity and uncertainty, and explore feedbacks between surface water, groundwater, and human systems. This work is organized in three distinct case studies: Chapters 2--3 explored the role of geologic heterogeneity, especially interconnected coarse-texture hydrofacies, on managed aquifer recharge processes in clastic sedimentary aquifer systems using a model that includes detailed representation of geologic heterogeneity and variably-saturated water flow physics. Chapter 4 compares water budgets for two regional-scale, coupled agricultural groundwater models in the Central Valley, and Chapter 5 is an opinion piece to advocate an incisive yet conceptually simple framework for transparent, real-time accounting of water stores and flows, including both groundwater and surface water, to inform California water markets organized around a central clearinghouse. 

Chapter 2 highlighted the role of subsurface geologic heterogeneity on recharge dynamics and identified important recharge phenomena that are not easily observed or simulated by typically-coarse resolution regional groundwater models. Results showed that a large (nearly 2 order-of-magnitude) range of cumulative recharge volumes between sites that is dependent primarily on the configuration of subsurface geologic facies. Results also suggest that while the majority of water volume and pressure response is transmitted through coarse-texture facies, the majority of the recharge volume is eventually stored in fine-texture facies, even for sites that have disproportionately large fractions of coarse-texture facies. This result suggests that fine-texture facies are the largest, albeit least accessible, reservoir for recharge in this system. This finding has important implications for aquifer conceptualization, because fine-texture facies are often considered as aquitards (or aquicludes) that do not appreciably participate as part of the overall aquifer system.

Using the aforementioned model from Chapter 2, Chapter 3 quantitatively evaluated the relative importance of hydrofacies configuration and hydrofacies hydraulic properties for managed aquifer recharge using novel sensitivity analysis techniques. This work comprised two fundamental components. First, exploratory simulations were performed at 100 randomly-sampled sites across the domain to to evaluate the correlation between geologic and hydrologic site characteristics and simulated recharge benefits. Results highlighted the value representing subsurface geologic configuration by upscaling vertical $K$, rather than solely identifying surficial soils that are favorable for recharge. Second, subsequent local and global sensitivity analyses showed that the geologic configuration was the most important factor when considering managed aquifer recharge potential in this system. To our knowledge, this study was the first of its kind to incorporate of a measure of geologic configuration with a geologic proxy parameter in formal sensitivity analyses. The combination of Chapters 2 and 3 help to (1) improve understanding the role of geologic heterogeneity on MAR processes and (2) provide insight into potential strategies to characterize subsurface geologic heterogeneity when considering recharge feasibility.

Chapter 4 compared water budget accounting methodologies for two regional-scale models that couple groundwater and agricultural water models in the Central Valley, California. Results showed that water budgets generally showed relatively poor agreement at sub-regional scales that are important for water management, especially in the northern Central Valley region. Improved agreement was observed at larger spatial scales, reflecting an error compensation effect. Systematic biases reflecting the partitioning of outflows from the landscape systems were observed between models, likely reflecting conceptual differences in the representation of the soil zone. These findings have important implications for the utility of these models for estimating water budgets for recently-passed groundwater legislation in California.

Chapter 5 presented an opinion on why a combination of comprehensive water resources accounting and a functioning water market that includes a central clearinghouse for water transactions holds great promise for improving water management in California. We contend that water management is often least effective when conducted piecemeal by many decentralized players with incomplete information, and that vague calls in the water science communities for greater interdisciplinarity as the answer have so far yielded only incremental progress in water management. Our opinion piece emphasized codependence of (1) water accounting to define the water stores and their interdependent changes in real time, and (2) water markets organized around a central clearinghouse that has proven so beneficial to the transparency of energy markets in places like California. 

While it focuses directly on the Central Valley of California, this dissertation presents tools and findings that are broadly applicable to other semi-arid, agriculturally-dominated, alluvial groundwater systems, many of which are facing similar challenges as those in California.