\section*{About this dissertation}   

The work that comprises this dissertation is the result of many fruitful collaborations across the University of California system and with other universities, along with myriad other federal, state, and local agencies and NGOs. Chapters 2--4 are my own work, and Chapter 5 was a collaborative effort of Ph.D. students in the CCWAS IGERT at UC Davis and Colorado School of Mines, in which I took a lead role. 

\section*{Notes on Chapter 2}

Chapter 2 was coauthored with Graham Fogg and Reed Maxwell and builds on the previous work of Casey Meirovitz and Yunjie Liu. The content emerged from discussions with all of the coauthors (and our colleagues) on the role of geologic heterogeneity on managed aquifer recharge (MAR) processes. All of the modeling was performed on Cheyenne, the National Center for Atmospheric Research's (NCAR) high-performance computer. This work has been accepted in \textit{Hydrogeology Journal} (doi: 10.1007/s10040-019-02033-9) and will likely appear in that journal in Fall 2019.
   
\section*{Notes on Chapter 3} 

Chapter 3 builds on the work from Chapter 1 and was coauthored with Laura Foglia, Graham Fogg, and Reed Maxwell. This work explores unanswered questions from the MAR simulations in Chapter 2 using novel sensitivity analysis techniques. Namely, we evaluate the relative importance of geologic architecture on MAR processes, as compared with other model parameters. All modeling was also performed on NCAR Cheyenne. This work is in review at \textit{Hydrologic and Earth System Science} (\textit{HESS}; preprint doi: 10.5194/hess-2019-412) and will likely appear in that journal in early 2020. 
    
\section*{Notes on Chapter 4} 

Chapter 4 compares the methodology for estimating water budgets, and especially unmeasured agricultural groundwater pumping, in the Central Valley of California. This work is coauthored by Graham Fogg, Laura Foglia, and Thomas Harter and emerged, in part, with the passage of the Sustainable Groundwater Management Act (SGMA) in 2014, which mandates that California better manage its groundwater resources. This work is in preparation for submission to \textit{Journal of Water Resources Planning and Management}. Portions of this research have also been contributed to a manuscript recently submitted to the \textit{Journal of Environmental Management}, for which I am a coauthor.

\section*{Notes on Chapter 5} 

Chapter 5 is an opinion piece which outlines how the combination of comprehensive water resources accounting and a functioning water market that includes a central clearinghouse for water transactions holds great promise for reducing magical thinking and dissolving many water resources management obstacles. I was was the lead author of this effort and was responsible for the bulk of the writing; however, I am deeply indebted to each of my coauthors, Ellen Bruno, Alejo Kraus-Polk, Stacy Roberts, and Lauren Foster, for their distinct perspectives and contributions to this piece. The concept for this manuscript emerged from the authors’ varied disciplines (hydrology, history, economics, geography) and from a conference that we and other CCWAS students developed and held in April, 2015 at the University of California, Davis. The conference, \textit{Water Scarcity in the West: Past, Present, Future}, brought together a diverse group of scholars to address the issue of water scarcity globally and through time. This work was published in a special issue of the journal \textit{Water} in late 2018 (doi: 10.3390/w10121720).