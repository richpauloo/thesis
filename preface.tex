\section*{About this dissertation}   

Groundwater resources are found in diverse settings worldwide, and like groundwater, the work herein is the product of many conversations and collaborations between researchers from the US, China, and Europe. 


\section*{Notes on Chapter 2}

Chapter 2 presents a data-driven methodology for estimating well failure applied to California's Central Valley. It was coauthored with Graham Fogg, Alvar Escriva-Bou, Helen Dahlke, Herve Guillon, and Amanda Fencl. The topic of statewide domestic well failure analysis arose after nearly one million digitized well completion reports detailing over 100 years of groundwater development in California were released to the public, permitting the first domestic well failure models. The initial kernel of this chapter won an award at the 2018 California Water Data Challenge, then grew to encompass a wider body of work. This work was published in \textit{Environmental Research Letters} (doi: 10.1088/1748-9326/ab6f10).

   
\section*{Notes on Chapter 3} 

Chapter 3 explores another consequence of aquifer depletion, and that is basin closure and subsequent salinization. Graham Fogg, Zhilin Guo and Thomas Harter were co-authors. This work was been submitted on 2020-07-07 to \textit{Journal of Hydrology} (preprint doi: 10.1002/essoar.10502733.1) and will likely appear in that journal by late 2020 or early 2021. 
    
    
\section*{Notes on Chapter 4} 

Chapter 4 builds on Chapter 3 by exploring the physical and hydrogeologic controls on subsurface contaminant transport that challenge the development of regional, upscaled transport models. It was coauthored with Graham Fogg, Zhilin Guo, and Christopher Henri. This work is in preparation for submission to \textit{Water Resources Research}. 

